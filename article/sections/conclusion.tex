\section{Conclusion}
In this paper we presented an approach that aims to reduce the complexity of static security analyzers. To do this, we build a generic AST (\astname{}) from the source code AST. The \astname{} is not bound to any language and can represent a large set of languages used in web applications. This way, we perform our analysis on the \astname{}, allowing us to decouple the source code parsing and the vulnerability detection. The solution was implemented in a tool called \toolname{}, using Java with parsers and tree walkers generated by ANTLR4. Our evaluation shows that \toolname{} can find vulnerabilities in different types of languages (e.g., statically and dynamically typed).