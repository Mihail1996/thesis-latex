%%%%%%%%%%%%%%%%%%%%%%%%%%%%%%%%%%%%%%%%%%%%%%%%%%%%%%%%%%%%%%%%%%%%%%%%
%                                                                      %
%     File: Thesis_Conclusions.tex                                     %
%     Tex Master: Thesis.tex                                           %
%                                                                      %
%     Author: Andre C. Marta                                           %
%     Last modified :  2 Jul 2015                                      %
%                                                                      %
%%%%%%%%%%%%%%%%%%%%%%%%%%%%%%%%%%%%%%%%%%%%%%%%%%%%%%%%%%%%%%%%%%%%%%%%

\chapter{Conclusions}
\label{chapter:conclusions}
In this work, we presented a new approach to static taint analysis that supports the addition of new languages with little programming effort. We were able to achieve this by taking advantage of the fact that the programming languages used in web applications have many similarities between them. With this in mind, instead of analyzing the source code directly, we first parse the source code and then build a generic AST (\astname{}) based on it. After that, we traverse the \astname{} to find vulnerabilities, thus decoupling the analysis from the parsing. The \astname{} does not represent every detail of a language, instead, it contains only what is needed to perform the analysis. This allows it to be able to represent a large set of programming languages used in web applications. The only parts bound to the language being analyzed are the parser, which in our implementation consists of generated code, and the module that converts source code AST into the \astname{}, which is usually less than 110 lines of code. 

The solution was implemented in the \toolname{} tool, using Java with parsers and tree walkers generated by ANTLR4. \toolname{} supports \implangs{} and was tested against several web applications written in different languages. Based on the results of our tests and the number of languages supported, we consider that our goal was successfully achieved.


\section{Future work}

The presented work leaves room for several possible improvements that were not possible to develop due to time constraints. \astname{} could be extended to support even more languages and more features (e.g., adding support for lambda functions). The taint analyzer could also be extended. For instance, we could make it more accurate by using a more precise pointer analysis. Furthermore, we could also track the value of each variable, removing this way the lambda limitation described in section \ref{limitations}. 